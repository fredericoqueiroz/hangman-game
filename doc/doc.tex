\documentclass[12pt]{article}
\usepackage[brazil]{babel}
\usepackage[utf8]{inputenc}
\usepackage{amsthm,amsfonts,amsmath}
\usepackage[a4paper,margin={1in,1in},vmargin={0.5in,0.5in}]{geometry}

\begin{document}
\title{%
    Documentação Trabalho Prático 1 \\
    \vspace{2em}
    \large DCC023 - Redes de Computadores \\
    UNIVERSIDADE FEDERAL DE MINAS GERAIS}

\author{Frederico Ribeiro Queiroz}
\maketitle

\section{Introdução}
O objetivo deste trabalho é implementar um jogo da forca simplificado, jogado entre cliente um
e um servidor, que se comunicam através de \emph{aplicações sockets}. \par
Em um jogo da forca, o jogador deve dar palpites até acertar qual a palavra proposta, sabendo a quantidade de letras presente na mesma.
Será utilizada uma versão simplificada onde os palpites errados não são contabilizados (o jogador não é `enforcado') e o jogo acaba quando todas as letras da palavra são encontradas. \par
Neste jogo, o cliente envia letras como palpites e o servidor recebe e responde os locais de ocorrência da letra, dada que ela exista na palavra. 
Os detalhes serão discutidos nas seções seguintes.

\section{Soluções implementadas}
Os programas foram implementados em \emph{linguagem C}. Para realizar a comunicação entre cliente e servidor, foram implementadas aplicações sockets que utilizam o protocolo TCP.
O servidor implementado suporta tanto endereços IPv4 quanto IPv6, através de uma constante fixada no código. \par
O esquema a seguir apresenta visualemente o funcionamento básico dos dois programas:



\end{document}